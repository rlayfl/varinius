\documentclass{article}
\usepackage{graphicx} % Required for inserting images

\title{Project Proposal: Understanding the Correlation between Simulation Variables and Stress in Players}
\author{Richard Lay-Flurrie}
\date{July 2025}

\begin{document}

\maketitle

\begin{abstract}
    Abstract
\end{abstract}

\section{Introduction}

The use of simulation to supplement training for stressful situations has been practice for a long time. However, high fidelity computer-based simulations are a relatively new concept which have been quickly adopted for training purposes. There is lot of overlap with the use of computer-based simulations for recreation and training with evidence of stress being caused by these simulations for a myriad of reasons. 

There appears to be a gap in knowledge concerning which specific variables can cause these feelings of stress in participants. 

\section{Literature Review}

Actually being able to measure stress levels in a participant is the first step. 



\section{Variables}

The experiment will consist of 8 binary variables, 4 of the form audio and visual (AV) stimuli and 4 identified decision stressors.

\subsection{Audio and Visual (AV) Stimuli}

AV stimuli will be used to simulate a naval navigation exercise. The AV stimuli will consist of the following:

\begin{itemize}
    \item The amount of light in the environment, which will be either bright or dark.
    \item The type of weather, which will be either clear or stormy.
    \item The presence of simulated sounds of people in distress
    \item The presence of vague horror noises, which will actually be recordings of deer and foxes.
\end{itemize}

\subsection{Decision Stressors}

Decision stressors will be used to simulate the stress of making decisions in a naval navigation exercise. The decision stressors will consist of the following:

\begin{itemize}
    \item Information Overload, the participant will be given a lot of information to remember
    \item Time Pressure, the participant will be given a time limit to make a decision
    \item Complexity, the participant will be given a complex scenario to navigate
    \item Competition, the participant will be competing for the best score against other participants
\end{itemize}






\section{Research Question}

What is the correlation between this project's chosen variables and the stress levels of participants in a simulated naval navigation exercise in Unreal Engine 5?

\section{Project Plan}

I expect this project to last for 4 weeks.

This would allow for 3 weeks of gathering data and 1 week for analysing the data and writing a report on it.

\section{Risk Assessment}

\subsection{Panic Attacks/Medical Emergencies}

It is possible that, while there is no risk of physical harm or loss to a participant, they may suffer a panic attack due to the focus on them and their performance under decision stressors. Furthermore, they may have an irrational fear of the GSR sensor, being worried that it might hurt them suddenly. To mitigate this risk, it is important that participants understand that the GSR sensor has been tested and will not hurt them. Furthermore, they should know that they can stop and leave at any time with no risk of penalty, punishment or judgement.

In the event a participant does suffer a panic attack, I should stay calm, offer them reassurance and seek professional assistance. In this case, the assistance would be from the university help desk staff.

In the event of any other medical emergency, the university help desk staff would be contacted. They have specifically asked us not to contact the emergency services directly and to go through them as they can guide emergency services to specific locations on campus efficiently.

\subsection{Equipment Failures}

The participant will be asked to wear a galvanic skin response (GSR) sensor and use a steering wheel controller. 

It is possible that a student could suffer an electric shock from the equipment, become tangled in cables or have the steering wheel fall on them.

To mitigate these risks, it is important that the equipment is tested properly by the University of Essex's School of Computer Science and Electronic Engineering IT team, cables are properly hidden away and secured where necessary and the steering wheel is properly fitted to the table. The volunteer will also be briefed on how to properly use the steering wheel. They will be instructed to sit down and not pull on the wheel as this could dislodge it from the table or desk.

\section{References}

\end{document}
